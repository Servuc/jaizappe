\documentclass[10pt,foldmark,notumble]{leaflet}

\usepackage[utf8]{inputenc}
\usepackage{graphicx}
\usepackage{listingsutf8}
\usepackage{color}
\usepackage[usenames,x11names]{xcolor}
\definecolor{citeGray}{HTML}{696969}
\definecolor{bulletlist}{HTML}{001020}
\definecolor{section}{HTML}{111111}
\definecolor{subsection}{HTML}{222222}
\definecolor{subsubsection}{HTML}{333333}
\usepackage{fourier}
\usepackage[top=0cm,bottom=0cm,left=0cm,right=0cm]{geometry}
\usepackage{caption}
\usepackage{makeidx}
\usepackage{amsmath,amsfonts,amssymb}
\usepackage{adjustbox}
\usepackage{blindtext}
\usepackage[babel=true]{csquotes}
\captionsetup{figurewithin=none}
\captionsetup{tablewithin=none}
\usepackage{changepage}

\def\changemargin#1#2#3#4{\list{}{\parsep#4\topsep#3\rightmargin#2\leftmargin#1}\item[]}
\let\endchangemargin=\endlist

\newcommand{\bmar}{\begin{changemargin}{0.2cm}{0.2cm}{-0.07cm}{0cm} }
\newcommand{\emar}{\end{changemargin}}

\lstset{literate=
  {á}{{\'a}}1 {é}{{\'e}}1 {í}{{\'i}}1 {ó}{{\'o}}1 {ú}{{\'u}}1
  {Á}{{\'A}}1 {É}{{\'E}}1 {Í}{{\'I}}1 {Ó}{{\'O}}1 {Ú}{{\'U}}1
  {à}{{\`a}}1 {è}{{\`e}}1 {ì}{{\`i}}1 {ò}{{\`o}}1 {ù}{{\`u}}1
  {À}{{\`A}}1 {È}{{\'E}}1 {Ì}{{\`I}}1 {Ò}{{\`O}}1 {Ù}{{\`U}}1
  {ä}{{\"a}}1 {ë}{{\"e}}1 {ï}{{\"i}}1 {ö}{{\"o}}1 {ü}{{\"u}}1
  {Ä}{{\"A}}1 {Ë}{{\"E}}1 {Ï}{{\"I}}1 {Ö}{{\"O}}1 {Ü}{{\"U}}1
  {â}{{\^a}}1 {ê}{{\^e}}1 {î}{{\^i}}1 {ô}{{\^o}}1 {û}{{\^u}}1
  {Â}{{\^A}}1 {Ê}{{\^E}}1 {Î}{{\^I}}1 {Ô}{{\^O}}1 {Û}{{\^U}}1
  {œ}{{\oe}}1 {Œ}{{\OE}}1 {æ}{{\ae}}1 {Æ}{{\AE}}1 {ß}{{\ss}}1
  {ç}{{\c c}}1 {Ç}{{\c C}}1 {ø}{{\o}}1 {å}{{\r a}}1 {Å}{{\r A}}1
  {€}{{\EUR}}1 {£}{{\pounds}}1
}

\usepackage{sectsty}
\sectionfont{\color{section}{}}
\subsectionfont{\color{subsection}{}}
\subsubsectionfont{\color{subsubsection}{}}
\definecolor{grisFonce}{HTML}{333333}
\definecolor{grisClair}{HTML}{dddddd}
\lstset{ %
  backgroundcolor=\color{grisClair},
  breaklines=true,
  language=[LaTeX]{TeX}
  }
\usepackage[colorlinks,urlcolor=grisFonce,linkcolor=grisFonce]{hyperref}
\usepackage{titlesec}
\titlespacing*{\section}
  {0.25cm}% decalage a gauche (positif ou negatif)
  {1ex}% espacement vertical avant
  {1ex}% espacement vertical apres
\titlespacing*{\subsection}
  {0.5cm}% decalage a gauche (positif ou negatif)
  {1ex}% espacement vertical avant
  {1ex}% espacement vertical apres
\titlespacing*{\subsubsection}
  {0.75cm}% decalage a gauche (positif ou negatif)
  {1ex}% espacement vertical avant
  {1ex}% espacement vertical apres



% modif 19 decembre 2003
\usepackage[francais]{babel}

%% modif du 25 novemebre 1999
%http://www.tug.dk/FontCatalogue/dejavusans/
%\usepackage[T1]{fontenc}
%\renewcommand*\familydefault{\sfdefault} %% Only if the base font of the document is to be sans serif
\usepackage[light]{kurier}
\usepackage[T1]{fontenc}

\setlength{\parindent}{0pt}
\setlength{\parskip}{0pt}

\title{\#Jaizappé ...}
\author{Servuc}
\date{2016}


\pdfinfo{%
  /Title    (\#Jaizappé ...)
  /Author   (Servuc)
  /Creator  (Servuc)
  /Producer (Servuc)
  /Subject  (Cours)
  /Keywords ()
}


\begin{document}
    \fcolorbox{black}{black}{
    \begin{minipage}{\linewidth}
        \begin{center}
            {\Huge{\color{white}\#Jaizappé ...\\... l'Octave}}
        \end{center}
    \end{minipage}}

    \section{Histoire}
        \textbf{Octave} est un projet \textbf{GNU}. Apparu au début des années 90. Ce projet est maintenu et développé par \textbf{John W. Eaton}.\\
        \textbf{Octave} peut exécuter des scripts MatLab, ainsi que du code C++ compilé ($^*.oct$). Un script Octave est suffixé en $.m$.
    \section{Bases}
        \subsection{Commentaire}
            Un commentaire est un message caché à l'exécution.
            \begin{lstlisting}[language=Octave]
%Ceci est un commentaire
            \end{lstlisting}
        \subsection{Afficher du texte}
            Pour afficher du texte ou des données, on dispose de \textbf{disp} (Le résultat est mis en commentaire et \textit{ans} signifie \textit{réponse}) :
            \begin{lstlisting}[language=Octave]
disp('Mon texte'); % Mon texte
maVariable = 4     % ans = 4
MAVARIABLE = 4     % ans = 4
c = 3 + 1          % ans = 4
disp(maVariable);  % 4
            \end{lstlisting}
        \subsection{Demander une valeur}
            \begin{lstlisting}[language=Octave]
a = input('Entrer la valeur de a :');
s = input('Entrer une chaine :', 's');
            \end{lstlisting}
        \subsection{Calculs basiques}
            \begin{lstlisting}[language=Octave]
a = 5 + 1; %addition
b = a * 3; %multiplication
c = b - 6; %soustraction
d = c / 1; %division
e = d ^ 3; %exposant
            \end{lstlisting}
            De même, on dispose de valeurs de référence :
            \begin{itemize}
                \item \textbf{pi} : $3.14$
                \item \textbf{i} : Imaginaire $\sqrt{-1} = 1$
                \item \textbf{e} : Nombre d'Euler
                \item \textbf{inf} : Infini
                \item \textbf{NaN} : Not a Number
            \end{itemize}
            Ainsi que des fonctions indispensables :
            \begin{center}
                \begin{tabular}{| l | c |}
                    \hline
                    Description & Commande \\
                    \hline
                    Modulo de $x$ par $y$ & mod(x,y); \\
                    Cos, Sin, Tan & cos(x); sin(x); tan(x);\\
                    Les arcs & acos(x); asin(x); atan(x); \\
                    Exp., Log et Log$_{10}$ & exp(x); log(x); et log10(x);  \\
                    Val. abs. et Racine & abs(x); sqrt(x);\\
                    Arrondi, $\nearrow$ et $\searrow$ & round(x); ceil(x); et floor(x);\\
                    Aléatoire & rand;\\
                    \hline
                \end{tabular}
            \end{center}
    \section{Vecteurs et matrices}
        \subsection{Déclaration}
            Déclarations manuelles des valeurs :
            \begin{lstlisting}[language=Octave]
%Un vecteur
v1 = [0 1 2 3];
v2 = 1:10   % debut:fin
v3 = 1:2:10 % debut:pas:fin
%Une matrice
m = [0 1;2 3];
            \end{lstlisting}
            Déclarations programmées :
            \begin{center}
                \begin{tabular}{| l | c |}
                    \hline
                    Description & Commande \\
                    \hline
                    Transposé de A et trace & A'; trace(A);\\
                    Mat. aléa. de $n\times n$ et $n\times m$ & rand(n); rand(n,m);\\
                    Matrice de 1 & ones(n); ou ones(n,m);\\
                    Matrice de 0 & zeros(n); ou zeros(n,m);\\
                    Mat. de 0 et diag. à 1 & eye(n);\\
                    Vect. de $a$ à $b$ à $n$ élém. & linspace(a, b, n);\\
                    Vect. de $10^a$ à $10^b$ à $n$ élém. & logspace(a, b, n);\\
                    Vect. et Mat. diag. de A & diag(A); et diag(diag(A));\\
                    \hline
                \end{tabular}
            \end{center}
        \subsection{Calculs basiques}
            On peut calculer élément par élément : $C_{ij} = A_{ij} \bigstar B_{ij}$, il suffit de préfixer $+$, $*$, $-$, $/$ et $\hat{ }$ d'un point :
            \begin{lstlisting}[language=Octave]
C = A .+ B
            \end{lstlisting}
            Opération entre deux matrices :
            \begin{lstlisting}[language=Octave]
C = A + B
            \end{lstlisting}
        \subsection{Manipulation de matrices}
            Modifier la taille d'une matrice, \textit{e.g.} $2\times 3 \rightarrow 1\times 6$ :
            \begin{lstlisting}[language=Octave]
A = rand(2,3);     % 2 lignes, 3 colonnes
B = reshape(A,1,6);% 1 ligne, 6 colonnes
            \end{lstlisting}
            Pour ordonner un vecteur :
            \begin{lstlisting}[language=Octave]
A = sort(A); % Tri dans l'ordre croissant
            \end{lstlisting}
            Pour faire des rotations matricielles :
            \begin{lstlisting}[language=Octave]
A = rot90(A);    % Rotation de 90 deg.
A = rot90(A, -2);% Rotation de -180 deg.
            \end{lstlisting}
            Récupérer des valeurs :
            \begin{lstlisting}[language=Octave]
B = A(2,3)    % Récupère la val. en 2 3
C = A(2,[1,3])% Récupère en 2 1 et 2 3
D = A(:,3)    % Récupère la colonne 3
E = A(2,:)    % Récupère la ligne 2
            \end{lstlisting}
            Ajouter des vecteurs et matrices ensemble,\\Par exemple on veut une matrice $5\times 5$ :
            \begin{lstlisting}[language=Octave]
A11 = rand(2,2);A12 = rand(2,3);
A21 = rand(3,2);A22 = rand(3,3);
A = [A11 A12; A21 A22];
            \end{lstlisting}
            Pour obtenir la matrice triangulaire ...
            \begin{lstlisting}[language=Octave]
B = tril(A);   % ... inférieur
C = triu(A);   % ... supérieur
[L,U] = lu(A); % Décomposition LU
[Q,R] = qr(A); % Décomposition QR
            \end{lstlisting}

            Inverse, déterminant d'une matrice carrée :
            \begin{lstlisting}[language=Octave]
AA = inv(A);
b = det(A);
            \end{lstlisting}
            Norme et conditionnement d'un vecteur ou d'une matrice :
            \begin{lstlisting}[language=Octave]
c = norm(A); 
d = cond(A); % norm(A) * norm(inv(A))
            \end{lstlisting}
            Valeurs propres et vecteur diagonale :
            \begin{lstlisting}[language=Octave]
[VectPro, Diag] = eig(A);
            \end{lstlisting}
    \section{Graphiques}
        Pour afficher un graphique 2D, on utilise :
        \begin{lstlisting}[language=Octave]
title('Mon titre');          %facultatif
label('label Y', 'label X'); %facultatif
plot(myData(:,1), myData(:,2));
figure; %sert à afficher la fenetre
        \end{lstlisting}
        Si l'on veut 2 séries de données :
        \begin{lstlisting}[language=Octave]
hold on %prends 2 séries à la fois
plot(myData(:,1), myData(:,2));
plot(myData(:,1), myData(:,3));
figure; 
        \end{lstlisting}
        On peut dessiner des grilles :
        \begin{lstlisting}[language=Octave]
[X,Y] = meshgrid(debut:pas:fin);
        \end{lstlisting}
        On peut personnaliser le graphique avec des options :
        \begin{center}
            \begin{tabular}{| c | c ||  c | c || c | c |}
                \hline
                Descri. & Arg. & Descri. & Arg. & Descri. & Arg.\\
                \hline
                Noir & k & Rouge & r & Vert & g\\
                Bleu & b & Magenta & m & Cyan & c\\
                \hline
                Croix & + x & Rond & o & Étoile & *\\
                Point & . & Carré & s & Diamant & d\\
                $\bigtriangleup$ $\bigtriangledown$ $\lhd$ $\rhd$ & $\hat{ }$ v $< >$ & Pent. & p & Hexa. & h\\
                \hline
            \end{tabular}
        \end{center}
        \begin{lstlisting}[language=Octave]
plot(myData(:,2), myData(:,3), 'ro');
figure;
        \end{lstlisting}
        On a aussi :
        \begin{itemize}
            \item bar(...); et barh(...); : Histogramme et Histo. Horizon.;
            \item semilogx(...); semilogy(...); : Log sur $X$ ou $Y$;
            \item loglog(...); : Log sur $X$ et $Y$;
        \end{itemize}
        Dessin d'une fonction sur pointeur :
        \begin{lstlisting}[language=Octave]
myFonc = @fonction; % e.g. @sin
fplot(myFonc, [limit1,limit2]); %3eme arg :
figure;                   %couleur et forme
        \end{lstlisting}
        \subsection{Sous-graphique}
            \begin{lstlisting}
subplot(1,2,1); %ligne, col, index
plot([0:10], sin([0:10]));
subplot(1,2,2); %ligne, col, index
plot([0:10], cos([0:10]));
figure;
            \end{lstlisting}

    \section{Polynômes}
        \subsection{Déclaration}
            Un polynôme se déclare par ses valeurs tel un vecteur,
            \textit{e.g. :} $-4x^4 -x^3 + 3x + 2$ :
            \begin{lstlisting}
p = [-1, 0, 3, 2];
polyout(p, 'x')
% ans = - 1*x^3 + 0*x^2 + 3*x^1 + 2
            \end{lstlisting}
            Racine, dérivée et intégration :
            \begin{lstlisting}
q1 = roots(p);
q2 = polyder(p);
q3 = polyint(p);
            \end{lstlisting}
            Multiplication et division polynômiale :
            \begin{lstlisting}
result = conv(p, q)
[b, r] = deconv(y, a) % y = ab + r
            \end{lstlisting}
    \section{Scripts}
        \subsection{Lire un fichier}
            Le fichier doit se présenter sous forme \textit{CSV} avec l'espace comme séparateur. On obtient une matrice :
            \begin{lstlisting}
myData = dlmread('donnees.txt');
            \end{lstlisting}
        \subsection{Conditions}
            \begin{lstlisting}
if (condition)
	%instructions
elseif (condition)
	%instructions
else
	%instructions
end
            \end{lstlisting}
        \subsection{Boucles}
            On a la boucle \textbf{for} qui prend en argument une série :
            \begin{lstlisting}
for cpt = debut:fin %Le pas est possible
	%instructions
end

for val = sort(rand(1,8));
	%instructions
end
            \end{lstlisting}
            La boucle \textbf{while} et \textbf{do-while}:
            \begin{lstlisting}
while (x < 5)
	%instructions
end

do
	%instructions
until (x < 5)
            \end{lstlisting}
            Bien sûr, les instructions \textbf{continue} et \textbf{break} fonctionnent.
        \subsection{Fonctions}
            Elles sont à définir au DÉBUT.\\
            Ci-dessous, \textit{valueX} sont des valeurs de retours et \textit{argX} les arguments.
            \begin{lstlisting}
function [value1[, value2, ...]]
= name( [arg1[, arg2, ...]] )
    value1 = ... % pour le retour
    return; % pour sortir de la fonction
endfunction
            \end{lstlisting}
            Il y a aussi les fonctions anonymes, qui sont des fonctions sans nom stockées dans une variable.
            \begin{lstlisting}
f = @(x) x*x % soit x au carré
f(2)         % retourne 4 
arrayfun(f, [2 3 4]) % appel du C
% ou Fortran dans le cas vecteur !
            \end{lstlisting}
    \section{Algèbres}
        Calcul d'une intégrale, il faut mettre la fonction en pointeur et renseigner les limites :
        \begin{lstlisting}[language=Octave]
myFonc = @uneFonction; % e.g. @sin
quad(myFonc,0,pi); % fonc, début, fin
        \end{lstlisting}
        Calcule d'une valeur :
        \begin{lstlisting}[language=Octave]
myFonc = @uneFonction; % e.g. @sin
feval(myFonc, arg); % fonc, valeur
%On peut aussi faire
myFonc(arg);
        \end{lstlisting}
        On peut analyser une fonction :
        \begin{lstlisting}[language=Octave]
s = input('Entrez une fonction :','s');
myFonc = inline(s);
        \end{lstlisting}
    \section{Équations différentielles}
        On a dans le cas de \textbf{Lorentz} :
        \begin{itemize}
            \item $\frac{dx}{dt} = a(y(t)-x(t))$
            \item $\frac{dy}{dt} = cx(t) - x(t)z(t) - y(t)$
            \item $\frac{dz}{dt} = x(t)y(t) - bz(t)$
        \end{itemize}

        \begin{lstlisting}[language=Octave]
function [f] = lorentz(u,t)
	a = 10; b = 8/3; c = 28;
	f(1) = a * (u(2) - u(1));
	f(2) = c * u(1) - u(1) * u(3) - u(2);
	f(3) = u(1) * u(2) - b * u(3);
endfunction
t = linspace(0,50,1001);
initVal = [1; 1; 1];
[myRes] = lsode('lorentz', initVal, t);
plot3(myRes(:,1), myRes(:,2), myRes(:,3));
figure;
        \end{lstlisting}


    \vfill
\fcolorbox{black}{black}{
\begin{minipage}{\linewidth}
{\color{white}\begin{center}\hfill http://github.com/Servuc/jaizappe \hfill\LaTeX\hfill Licence GPLv3\hfill $\,$\end{center}}
\end{minipage}}
\end{document}