\documentclass[10pt,foldmark,notumble]{leaflet}

\usepackage[utf8]{inputenc}
\usepackage{graphicx}
\usepackage{listingsutf8}
\usepackage{color}
\usepackage[usenames,x11names]{xcolor}
\definecolor{citeGray}{HTML}{696969}
\definecolor{bulletlist}{HTML}{001020}
\definecolor{section}{HTML}{111111}
\definecolor{subsection}{HTML}{222222}
\definecolor{subsubsection}{HTML}{333333}
\usepackage{fourier}
\usepackage[top=0cm,bottom=0cm,left=0cm,right=0cm]{geometry}
\usepackage{caption}
\usepackage{makeidx}
\usepackage{amsmath,amsfonts,amssymb}
\usepackage{adjustbox}
\usepackage{blindtext}
\usepackage[babel=true]{csquotes}
\captionsetup{figurewithin=none}
\captionsetup{tablewithin=none}
\usepackage{changepage}

\def\changemargin#1#2#3#4{\list{}{\parsep#4\topsep#3\rightmargin#2\leftmargin#1}\item[]}
\let\endchangemargin=\endlist

\newcommand{\bmar}{\begin{changemargin}{0.2cm}{0.2cm}{-0.07cm}{0cm} }
\newcommand{\emar}{\end{changemargin}}

\lstset{literate=
  {á}{{\'a}}1 {é}{{\'e}}1 {í}{{\'i}}1 {ó}{{\'o}}1 {ú}{{\'u}}1
  {Á}{{\'A}}1 {É}{{\'E}}1 {Í}{{\'I}}1 {Ó}{{\'O}}1 {Ú}{{\'U}}1
  {à}{{\`a}}1 {è}{{\`e}}1 {ì}{{\`i}}1 {ò}{{\`o}}1 {ù}{{\`u}}1
  {À}{{\`A}}1 {È}{{\'E}}1 {Ì}{{\`I}}1 {Ò}{{\`O}}1 {Ù}{{\`U}}1
  {ä}{{\"a}}1 {ë}{{\"e}}1 {ï}{{\"i}}1 {ö}{{\"o}}1 {ü}{{\"u}}1
  {Ä}{{\"A}}1 {Ë}{{\"E}}1 {Ï}{{\"I}}1 {Ö}{{\"O}}1 {Ü}{{\"U}}1
  {â}{{\^a}}1 {ê}{{\^e}}1 {î}{{\^i}}1 {ô}{{\^o}}1 {û}{{\^u}}1
  {Â}{{\^A}}1 {Ê}{{\^E}}1 {Î}{{\^I}}1 {Ô}{{\^O}}1 {Û}{{\^U}}1
  {œ}{{\oe}}1 {Œ}{{\OE}}1 {æ}{{\ae}}1 {Æ}{{\AE}}1 {ß}{{\ss}}1
  {ç}{{\c c}}1 {Ç}{{\c C}}1 {ø}{{\o}}1 {å}{{\r a}}1 {Å}{{\r A}}1
  {€}{{\EUR}}1 {£}{{\pounds}}1
}

\usepackage{sectsty}
\sectionfont{\color{section}{}}
\subsectionfont{\color{subsection}{}}
\subsubsectionfont{\color{subsubsection}{}}
\definecolor{grisFonce}{HTML}{333333}
\definecolor{grisClair}{HTML}{dddddd}
\lstset{ %
  backgroundcolor=\color{grisClair},
  breaklines=true,
  language=[LaTeX]{TeX}
  }
\usepackage[colorlinks,urlcolor=grisFonce,linkcolor=grisFonce]{hyperref}
\usepackage{titlesec}
\titlespacing*{\section}
  {0.25cm}% decalage a gauche (positif ou negatif)
  {1ex}% espacement vertical avant
  {1ex}% espacement vertical apres
\titlespacing*{\subsection}
  {0.5cm}% decalage a gauche (positif ou negatif)
  {1ex}% espacement vertical avant
  {1ex}% espacement vertical apres
\titlespacing*{\subsubsection}
  {0.75cm}% decalage a gauche (positif ou negatif)
  {1ex}% espacement vertical avant
  {1ex}% espacement vertical apres



% modif 19 decembre 2003
\usepackage[francais]{babel}

%% modif du 25 novemebre 1999
%http://www.tug.dk/FontCatalogue/dejavusans/
%\usepackage[T1]{fontenc}
%\renewcommand*\familydefault{\sfdefault} %% Only if the base font of the document is to be sans serif
\usepackage[light]{kurier}
\usepackage[T1]{fontenc}

\setlength{\parindent}{0pt}
\setlength{\parskip}{0pt}

\title{\#Jaizappé ...}
\author{Servuc}
\date{2016}


\pdfinfo{%
  /Title    (\#Jaizappé ...)
  /Author   (Servuc)
  /Creator  (Servuc)
  /Producer (Servuc)
  /Subject  (Cours)
  /Keywords ()
}


\begin{document}
    \fcolorbox{black}{black}{
    \begin{minipage}{\linewidth}
        \begin{center}
            {\Huge{\color{white}\#Jaizappé ...\\... le \LaTeX}}
        \end{center}
    \end{minipage}}
    \section{Histoire}
        Le \LaTeX ([lat$\epsilon$k]) est un langage de création de documents, créé en 1983 par \textit{Leslie Lamport}. Les fichiers sont en \textit{.tex}.
    \section{Bases}
        \subsection{Code minimal}
            \begin{lstlisting}
\documentclass{article}
\usepackage[french]{babel} % Français
\usepackage[utf8]{inputenc}  % Accents
% En-tête, Ceci est un commentaire
\begin{document}
  % Contenu du document
\end{document}
            \end{lstlisting}
        \subsection{Classes de documents}
            Définit comment le document sera formé à sa compilation.
            \begin{itemize}
                \item \textbf{article} : Petit document sans chapitres;
                \item \textbf{report} : Document moyen, ajout de \textit{\textbackslash chapter};
                \item \textbf{book} et \textbf{memoir} : Idem que \textit{report};
                \item \textbf{letter} : Pour le courrier;
                \item \textbf{IEEEtran} : Pour les articles formatés pour les IEEE;
                \item \textbf{slider} et \textbf{beamer} : Présentations (non traités ici);
                \item \textbf{leaflet} : Flyer (comme ce document);
                \item \textbf{minimal} : Le strict minimum (peu pratique).
            \end{itemize}
            La classe peut être précédée d'options :
            \begin{itemize}
                \item La taille de la police : \textbf{10pt}, \textbf{11pt} et \textbf{12pt};
                \item Le type de papier : \textbf{a4paper}, \textbf{a5paper}, \textbf{letterpaper}, \textbf{b5paper}, \textbf{executivepaper} et \textbf{legalpaper};
                \item Recto-verso : \textbf{twoside}, ou non \textbf{oneside};
                \item Paysage : \textbf{landscape};
                \item Deux colonnes : \textbf{twocolumn};
                \item Formules (maths) : Placées à gauche : \textbf{fleqn};
                \item Page de présentation : \textbf{titlepage} ou non \textbf{notitlepage};
                \item Début de chapitre : à droite \textbf{openright}, ou \textbf{openany}.
            \end{itemize}
            On utilise les options ainsi par exemple :
            \begin{lstlisting}
\documentclass[12pt,openany,landscape]{book}
            \end{lstlisting}
    \section{En-tête}
        \emph{Ici, on définit l'aspect graphique du document. On les place entre le \textbackslash documentclass\{\} et le \textbackslash begin\{document\}.}
        \subsection{Les couleurs}
            \begin{lstlisting}
\usepackage[usenames,x11names]{xcolor}
            \end{lstlisting}
            Il suffit d'aller sur \textit{Google} et de taper \textit{latex x11names}.\\
            Si la liste est trop courte, vous pouvez en créer :
            \begin{lstlisting}
\definecolor{nom1}{rgb}{1,0.5,0} % De 0 à 1
\definecolor{nom2}{RGB}{255,128,0} % 0 à 255
\definecolor{nom3}{HTML}{FF8000} % De 0 à F
            \end{lstlisting}
            Vous pouvez :
            \begin{lstlisting}
\pagecolor{couleur} % Fond de page
\textcolor{couleur}{texte} % Texte colorer
{\color{couleur}texte} % Identique
\colorbox{couleur}{texte} % Fond du texte
\colorbox{coul1}{\color{coul2}text} % Les 2
            \end{lstlisting}
        \subsection{Polices d'écriture}
            Voir : \url{http://www.tug.dk/FontCatalogue/}.\\Il suffit d'ajouter une ou deux lignes en général dans l'en-tête.\\
            Si la police possède des variantes, il faut parfois indiquer le style après le \textit{\textbackslash begin\{document\}}.\\
            Les éléments de structure ne sont pas impactés ! Exemple :
            \begin{lstlisting}
\documentclass{article}
  % On indique la police (2 lignes)
  \usepackage{emerald}
  \usepackage[T1]{fontenc}
\begin{document}
  \ECFAugie % On indique le formattage
  Mon texte stylisé
\end{document}
            \end{lstlisting}
        \subsection{Styliser les éléments de structure}
            Il faut utiliser le package \textbf{titlesec} et faire :
            \begin{lstlisting}
\usepackage{titlesec}
\titleformat{\chapter}{\LARGE\bfseries}{Chapitre : \thechapter}
% ou \titleformat{\chapter}{\LARGE\bfseries}
%\titleformat{structure}{apparence}{label}
            \end{lstlisting}
            Les \textit{structure}s et l'\textit{apparence} sont définis par la suite.\\Pour le \textit{label}, il s'agit de l'aspect textuel. Il est facultatif. Pour appeler le nom de la structure, on utilise : \textbf{\textbackslash thestructure}.

        \subsection{Méta-informations}
            Il s'agit du titre, de l'auteur et de la date du document :
            \begin{lstlisting}
\title{Mon titre} \author{Moi !}
\thanks{Merci à ...} \date{2016}
            \end{lstlisting}
            Par défaut, la date indiquera le jour, le mois et l'année (\textbf{\textbackslash today}). Suivant la classe, il faudra ajouter la page de titre : \textbf{\textbackslash maketitle}.
        \subsection{Méta-informations des pages}
            \begin{lstlisting}
\usepackage{fancyhdr,lastpage}
\fancyhead[L]{06/2014}\fancyfoot[R]{Moi}
\fancyfoot[C]{\thepage/\pageref{LastPage}}
            \end{lstlisting}
            \textbf{\textbackslash fancyhead} pour le haut et \textbf{\textbackslash fancyhead} pour le bas.\\
            \textbf{L} : Gauche, \textbf{C} : Center, \textbf{R} : Droite.

        \subsection{Marges}
            \begin{lstlisting}
\usepackage[top=XX.XXcm,bottom=XX.XXcm,
   left=XX.XXcm,right=XX.XXcm]{geometry}
            \end{lstlisting}
            On peut mettre une seule option, ou plusieurs !


    \section{Éléments graphiques}
        \subsection{Éléments simples}
            \begin{lstlisting}
\LaTeX % Logo Latex
\newpage % Nouvelle page
\footnote{Ma note} % Note de bas de page
\textbf{Texte} ou {\bfseries Texte} % Gras
\textit{Texte} ou {\itshape Texte} %Italique
\textsc{Texte} ou {\scshape Texte} %Captital
\uppercase{Texte} % Majuscule
\underline{Texte} % Souligné
{\centering Texte} % Centre le texte ou élem
\begin{center}Texte\end{center} % Idem
            \end{lstlisting}
            Tailles d'écritures (\textit{\{\textbackslash taille texte\}} ou \textit{\textbackslash taille texte}) :
            \begin{itemize}
                \item Petit : \textbf{\textbackslash tiny}, \textbf{\textbackslash scriptsize}, \textbf{\textbackslash footnotesize};
                \item Moyen : \textbf{\textbackslash small}, \textbf{\textbackslash normalsize}, \textbf{\textbackslash large};
                \item Grand : \textbf{\textbackslash Large}, \textbf{\textbackslash LARGE}, \textbf{\textbackslash huge}, \textbf{\textbackslash HUGE}.
            \end{itemize}
        \subsection{Les structures}
            Inutilisable dans les \textit{letter} ! De la plus à la moins importante :
            \begin{enumerate}
                \setcounter{enumi}{-2}
                \item \textbf{\textbackslash part} : Les parties (facultatif);
                \item \textbf{\textbackslash chapter} : Pour les \textit{book}, \textit{memoir} et \textit{report};
                \item \textbf{\textbackslash section}, \textbf{\textbackslash subsection}, \textbf{\textbackslash subsubsection};
                \item \textbf{\textbackslash paragraph}, \textbf{\textbackslash subparagraph}.
            \end{enumerate}
            Un \textbf{paragraph}, \textbf{chapter} et \textbf{section} ne nécéssitent pas un élément plus important. On les utilise ainsi :
            \begin{lstlisting}
\section{Mon gros titre}
\subsection{Mon sous-titre}
\paragraph*{Mon paragraphe}
            \end{lstlisting}
            Pour afficher le sommaire dans le document :
            \begin{lstlisting}
\tableofcontents % 2 COMPILATIONS REQUISES
            \end{lstlisting}
            On peut éviter la numérotation avec $^*$ (Voir ci-dessus) mais il n'est pas dans le sommaire, sauf si l'on met ceci avant l'élément :
            \begin{lstlisting}
\addcontentsline{toc}{STRUCTURE_TYPE}{LABEL}
\addcontentsline{toc}{chapter}{Mon chapitre}
            \end{lstlisting}
        \subsection{Créer des espaces}
            \subsubsection{Basiques}
                Pour aller à la ligne : \textbf{\textbackslash\textbackslash}.\\
                Pour aller en bas d'une page : \textbf{\textbackslash vfill}.\\
                Pour aller à droite : \textbf{\textbackslash hfill}.
                Espace vertical : \textbf{\textbackslash vspace\{XX unité\}}.\\
                Espace horizontal : \textbf{\textbackslash hspace\{XX unité\}}.\\
                \textbf{XX} est un nombre entier et \textbf{unité} en \textit{cm}, \textit{mm}, \textit{pt} (point) ou \textit{in} (inch).
            \subsubsection{Alinéas}
                \begin{lstlisting}
\setlength{\parindent}{1cm} % Alinéa de 1cm
\noindent % Pas d'alinéa
                \end{lstlisting}
        \subsection{Les images}
            \begin{lstlisting}
\usepackage{graphicx}
            \end{lstlisting}
            La commande est : \textit{\textbackslash includesgraphics[options]\{image.format\}}.\\
            L'image doit être à côté du document en \textit{.tex}.\\
            Les options sont séparées par une virgule et facultatives :
            \begin{itemize}
                \item \textbf{width=..px} et \textbf{height=..px} : Ratio non respecté;
                \item \textbf{scale=..} : Ratio respecté;
                \item \textbf{angle=..} : Angle en degré.
            \end{itemize}
            On peut les accompagnées d'un titre :
            \begin{lstlisting}
\begin{figure}[p]
    \includegraphics[options]{image.png}
    \caption{Mon image}
    \label{fig:mon_image}
\end{figure}
            \end{lstlisting}
        \subsection{Les liens}
            \subsubsection{Lien dans le document}
                Pour référencer un élément du document sur une autre page:
                \begin{lstlisting}
\label{mon-marqueur} % Marqueur
\ref{mon-marqueur} % Lien vers le marqueur
\pageref{mon-marqueur} % Numéro de page
                \end{lstlisting}
            \subsubsection{Lien vers internet}
                \begin{lstlisting}
\usepackage[colorlinks,urlcolor=c1,
  linkcolor=c2]{hyperref}
                \end{lstlisting}
                \textbf{c1} et \textbf{c2} sont des NOMS de couleurs ! Par exemple \textit{red}, \textit{blue} ou personnel. (Options facultatives). Usage :
                \begin{lstlisting}
\url{http://} ou \href{http://}{Nom page}
\href{mailto:mail@domain.org}{Nom mail}
                \end{lstlisting}
        \subsection{Codes sources}
            \begin{lstlisting}
\usepackage{listingsutf8}
            \end{lstlisting}
            Usage :\\
            \colorbox{grisClair}{\textbf{\textbackslash begin\{lstlisting\}[options]\\Mon code source\\\textbackslash end\{lstlisting\}}}\\
            Les options sont variées :
            \begin{itemize}
                \item \textbf{language=xx} : C, TeX, PHP, HTML, etc. (Voir web);
                \item \textbf{breaklines=true|false} : Aller à la ligne;
                \item \textbf{keywordstyle=xx} et \textbf{commentstyle=xx} : Style des mots clés et des commentaires : En gras, couleurs, etc;
                \item \textbf{otherkeywords=\{*,mot,mot2\}} : Autres mots clés;
                \item \textbf{numbers=left|right|none} : Numéros de ligne;
                \item \textbf{backgroundcolor=\textbackslash color\{xx\}} : Fond;
                \item \textbf{stringstyle=xx} : Style des chaines de caractères;
                \item Et d'autres !
            \end{itemize}
            On peut définir un style général dans l'en-tête :
            \begin{lstlisting}
\lstset{ background=\color{blue},
  language={LaTeX}[TeX] }
            \end{lstlisting}
            Ou bien un style particulier dans l'en-tête:
            \begin{lstlisting}
\lstdefinestyle{myStyle}{
  backgroundcolor=\color{yellow} }
            \end{lstlisting}
            Et on l'utilise ainsi dans les options en mettant \textbf{style=myStyle}. De plus on peut include un code source externe :
            \begin{lstlisting}
\lstinputlisting[options]{monSource.format}
            \end{lstlisting}
        \subsection{Les listes}
            \subsubsection{Listes numérotées}
            \begin{lstlisting}
\begin{enumerate}
  \item Texte ou autre liste item. ou enum.
\end{enumerate}
            \end{lstlisting}
            On peut modifier le numéro actuel avec (à côté de \textbf{\textbackslash item}) :
            \begin{lstlisting}
\setcounter{enumi}{NOMBRE POS. OU NÉG.}
            \end{lstlisting}
            \subsubsection{Listes non-numérotées}
                Remplacer \textbf{enumerate} par \textbf{itemize}.
        \subsection{Les tableaux}
            Un tableau est défini ainsi :
            \begin{lstlisting}
\begin{tabular}{DETAIL_COLONNES}
  \hline   CONTENU_LIGNE \\
  \hline   CONTENU_LIGNE \\   \hline
\end{tabular}
            \end{lstlisting}
            Les \textbf{\textbackslash hline} permettent de faire une ligne horizontale.\\
            \textit{DETAIL\_COLONNES} est composé de $l, c, r, p\{Xcm\}, |, ||$, respectivement \textit{left}, \textit{center} et \textit{right} : ils indiquent la position du texte dans la colonne. $p\{Xcm\}$ permet de forcer le texte à aller à la ligne. $|$ et $||$ indiquent qu'il y a une ou deux lignes verticales.\\
            \textit{CONTENU\_LIGNE} est composé de TOUTES vos colonnes, séparées juste par \textbf{\&}.
            \begin{lstlisting}
\begin{tabular}{| l r || l r |}
  \hline
    $a$ & 1 & $\&$ & 2 \\
  \hline
\end{tabular}
            \end{lstlisting}
            Entourer d'un \textbf{\textbackslash begin\{center\} TABLEAU \textbackslash end\{center\}} votre tableau le rendra plus agréable.\\
            La fusion d'une case se fait avec (à la place du texte) :
            \begin{lstlisting}
\multicolumn{NB_COLONNE}{X}{TEXTE}
            \end{lstlisting}
            \textbf{X} correspond à $l, c, r, p\{Xcm\}, |$ et ou $||$, on peut mettre UNE lettre seulement ! \textbf{NB\_COLONNE} = colonnes remplacées.
        \subsection{Échapper les caractères (IMPORTANT)}
            Il faut les échapper sinon la compilation échoue !!!
            \begin{center}
                \begin{tabular}{| l r || l r || l r |}
                    \hline
                    $\%$ & \textbackslash \% & $\&$ & \textbackslash \& & $\$$ & \textbackslash \$\\
                    \hline
                    $\#$ & \textbackslash \# & $\{$ & \textbackslash \{ & $\}$ & \textbackslash \}\\
                    \hline
                    $\_$ & \textbackslash \_ & $\textbackslash$ & \multicolumn{3}{r|}{\textbackslash textbackslash} \\
                    \hline
                \end{tabular}
            \end{center}
        \subsection{Page de garde}
            La page de garde est définie par :
            \begin{lstlisting}
\begin{document}
  \begin{titlepage}
    {\HUGE Votre page \today}
  \end{titlepage}
  ... % Votre document
\end{document}
            \end{lstlisting}
            Il suffit de "\textit{créer des espaces basiques}" pour mettre en page.

\vfill
\fcolorbox{black}{black}{
\begin{minipage}{\linewidth}
{\color{white}\begin{center}\hfill http://github.com/Servuc/jaizappe \hfill\LaTeX\hfill Licence GPLv3\hfill $\,$\end{center}}
\end{minipage}}
\end{document}
