\documentclass[10pt,foldmark,notumble]{leaflet}

\usepackage[utf8]{inputenc}
\usepackage{graphicx}
\usepackage{listingsutf8}
\usepackage{color}
\usepackage[usenames,x11names]{xcolor}
\definecolor{citeGray}{HTML}{696969}
\definecolor{bulletlist}{HTML}{001020}
\definecolor{section}{HTML}{111111}
\definecolor{subsection}{HTML}{222222}
\definecolor{subsubsection}{HTML}{333333}
\usepackage{fourier}
\usepackage[top=0cm,bottom=0cm,left=0cm,right=0cm]{geometry}
\usepackage{caption}
\usepackage{makeidx}
\usepackage{amsmath,amsfonts,amssymb}
\usepackage{adjustbox}
\usepackage{blindtext}
\usepackage[babel=true]{csquotes}
\captionsetup{figurewithin=none}
\captionsetup{tablewithin=none}
\usepackage{changepage}

\def\changemargin#1#2#3#4{\list{}{\parsep#4\topsep#3\rightmargin#2\leftmargin#1}\item[]}
\let\endchangemargin=\endlist

\newcommand{\bmar}{\begin{changemargin}{0.2cm}{0.2cm}{-0.07cm}{0cm} }
\newcommand{\emar}{\end{changemargin}}

\lstset{literate=
  {á}{{\'a}}1 {é}{{\'e}}1 {í}{{\'i}}1 {ó}{{\'o}}1 {ú}{{\'u}}1
  {Á}{{\'A}}1 {É}{{\'E}}1 {Í}{{\'I}}1 {Ó}{{\'O}}1 {Ú}{{\'U}}1
  {à}{{\`a}}1 {è}{{\`e}}1 {ì}{{\`i}}1 {ò}{{\`o}}1 {ù}{{\`u}}1
  {À}{{\`A}}1 {È}{{\'E}}1 {Ì}{{\`I}}1 {Ò}{{\`O}}1 {Ù}{{\`U}}1
  {ä}{{\"a}}1 {ë}{{\"e}}1 {ï}{{\"i}}1 {ö}{{\"o}}1 {ü}{{\"u}}1
  {Ä}{{\"A}}1 {Ë}{{\"E}}1 {Ï}{{\"I}}1 {Ö}{{\"O}}1 {Ü}{{\"U}}1
  {â}{{\^a}}1 {ê}{{\^e}}1 {î}{{\^i}}1 {ô}{{\^o}}1 {û}{{\^u}}1
  {Â}{{\^A}}1 {Ê}{{\^E}}1 {Î}{{\^I}}1 {Ô}{{\^O}}1 {Û}{{\^U}}1
  {œ}{{\oe}}1 {Œ}{{\OE}}1 {æ}{{\ae}}1 {Æ}{{\AE}}1 {ß}{{\ss}}1
  {ç}{{\c c}}1 {Ç}{{\c C}}1 {ø}{{\o}}1 {å}{{\r a}}1 {Å}{{\r A}}1
  {€}{{\EUR}}1 {£}{{\pounds}}1
}

\usepackage{sectsty}
\sectionfont{\color{section}{}}
\subsectionfont{\color{subsection}{}}
\subsubsectionfont{\color{subsubsection}{}}
\definecolor{grisFonce}{HTML}{333333}
\definecolor{grisClair}{HTML}{dddddd}
\lstset{ %
  backgroundcolor=\color{grisClair},
  breaklines=true,
  language=[LaTeX]{TeX}
  }
\usepackage[colorlinks,urlcolor=grisFonce,linkcolor=grisFonce]{hyperref}
\usepackage{titlesec}
\titlespacing*{\section}
  {0.25cm}% decalage a gauche (positif ou negatif)
  {1ex}% espacement vertical avant
  {1ex}% espacement vertical apres
\titlespacing*{\subsection}
  {0.5cm}% decalage a gauche (positif ou negatif)
  {1ex}% espacement vertical avant
  {1ex}% espacement vertical apres
\titlespacing*{\subsubsection}
  {0.75cm}% decalage a gauche (positif ou negatif)
  {1ex}% espacement vertical avant
  {1ex}% espacement vertical apres



% modif 19 decembre 2003
\usepackage[francais]{babel}

%% modif du 25 novemebre 1999
%http://www.tug.dk/FontCatalogue/dejavusans/
%\usepackage[T1]{fontenc}
%\renewcommand*\familydefault{\sfdefault} %% Only if the base font of the document is to be sans serif
\usepackage[light]{kurier}
\usepackage[T1]{fontenc}

\setlength{\parindent}{0pt}
\setlength{\parskip}{0pt}

\title{\#Jaizappé ...}
\author{Servuc}
\date{2016}


\pdfinfo{%
  /Title    (\#Jaizappé ...)
  /Author   (Servuc)
  /Creator  (Servuc)
  /Producer (Servuc)
  /Subject  (Cours)
  /Keywords ()
}

\begin{document}
    \fcolorbox{black}{black}{
    \begin{minipage}{\linewidth}
        \begin{center}
            {\Huge{\color{white}\#Jaizappé ...\\... le XML et XSchema}}
        \end{center}
    \end{minipage}}

    \section{Histoire}
        La première version de \textbf{XML} (\textit{1.0}) est apparu en février 1998. Le \textbf{XML} est inspiré du \textbf{SGML} et \textbf{HTML}.\\
        \textbf{XML} signifie e\textbf{X}tensible \textbf{M}arkup \textbf{L}anguage.

        \subsection{Commentaires}
            Un commentaire est un message caché à l'exécution.
            \begin{lstlisting}[language=XML]
<!-- Un com. de 1 ou + ligne(s) -->
            \end{lstlisting}
    \section{Base}
        Un fichier \textbf{XML} a pour extension \textit{.xml} et commence par :
        \begin{lstlisting}[language=XML]
<?xml version="1.0">
        \end{lstlisting}
        Ensuite on ajoute des éléments :
        \begin{lstlisting}[language=XML]
<?xml version="1.0">
<!DOCTYPE livreDef>
<livre>
	<pages>
		<page numero="1">
			Première page
		</page>
		<page numero="2">
			Seconde page
		</page>
	</pages>
	<auteur>Michel</auteur>
	<isbn value="xXXXXXXXx" />
</livre>
        \end{lstlisting}
        Ici, on a décrit un livre. \textit{numero=``X''} est un attribut de \textit{page}.\\
        \textit{<livre>} est une balise ouvrante, \textit{</livre>} en est une fermante.\\
        \textit{Michel} est le contenu de \textit{auteur}.
        \subsection{Appel externe}
            On peut appeler une application externe lors de l'affichage
            \begin{lstlisting}[language=XML]
<?application code ?>
<!-- Exemple -->
<?php echo "Hello World"; ?>
            \end{lstlisting}
    \section{DTD}
        La \textbf{DTD} permet de formaliser un fichier \textit{XML}, elle se place juste après la ligne : \textit{<?xml ...>}.
        \subsection{Élément simple}
            C'est un élément sans balise dans son contenu :
            \begin{lstlisting}[language=XML]
<!ELEMENT page(#PCDATA)>
            \end{lstlisting}
        \subsection{Élément vide}
            C'est un élément sans contenu ni balise :
            \begin{lstlisting}[language=XML]
<!ELEMENT isbn EMPTY>
<!ATTLIST isbn value(CDATA)>
            \end{lstlisting}
        \subsection{Élément d'élément(s)}
            C'est un élément avec QUE des balises :
            \begin{lstlisting}[language=XML]
<!ELEMENT livre(pages, auteur+)>
<!ELEMENT pages(page+)>
            \end{lstlisting}
            Le \textbf{+} après \textit{page} signifie que l'on a au moins un élément \textit{page} : $[1, +\infty[$. $* = [0, +\infty[$, $? = [0,1]$. Si on ne met rien, cela veut dire 1.
        \subsection{Élément d'élément(s) ou simple}
            Admettons que l'on est une \textit{page} seulement, on peut directement mettre le contenu
            \begin{lstlisting}[language=XML]
<!ELEMENT pages(#PCDATA|page+)>
            \end{lstlisting}
            \begin{lstlisting}[language=XML]
<?xml version="1.0">
<!DOCTYPE livreDef>
<livre>
	<pages>Page 1</pages>
	<auteur>Michel</auteur>
</livre>
            \end{lstlisting}
        \subsection{Entités}
            Elles servent de raccourci :
            \begin{lstlisting}[language=XML]
<!ENTITY cle="valeur">
<!-- Et dans le XML -->
<noeud attr="&autreCle;">&cle;</noeud>
            \end{lstlisting}
        \subsection{Attribut d'élément}
            Équivalent à des paramètres :
            \begin{lstlisting}[language=XML]
<!ATTLIST balise nomAttri (type) option>
<!-- exemple -->
<!ATTLIST page numero (CDATA) #REQUIRED>
            \end{lstlisting}
            Les types sont :
            \begin{itemize}
                \item \textbf{CDATA} : Une chaine de caractères;
                \item \textbf{ID} : Attribut unique;
                \item \textbf{IDREF} : Fait référence à un \textbf{ID};
                \item \textbf{valeur1|valeur2|...} : On met les possibilités;
                \item \textbf{ENTITY} : Fait référence à une entitée.
            \end{itemize}
            Les options sont :
            \begin{itemize}
                \item \textbf{\#REQUIRED} : Attribut obligatoire;
                \item \textbf{\#IMPLIED} : (Par défaut), attribut non obligatoire;
                \item \textbf{\#FIXED valeur} : Indique une valeur par défaut.
            \end{itemize}
        \subsection{Doctype}
            \begin{lstlisting}[language=XML]
<!DOCTYPE livreDef [ <!-- DTD ici --> ]>
            \end{lstlisting}
            Ensuite, on met le \textbf{XML}. On peut l'inclure par un lien système (Ensuite on met le \textbf{XML}) :
            \begin{lstlisting}[language=XML]
<!DOCTYPE livreDef SYSTEM "livreDef.dtd">
            \end{lstlisting}
        \subsection{Fichiers externes}
            Il s'agit d'images, sons, documents, ... :
            \begin{lstlisting}[language=XML]
<!DOCTYPE vrml [
  <!ELEMENT vrml (#PCDATA)>
  <!NOTATION vrml PUBLIC "VRML 1.0">
]>
<!DOCTYPE img [
  <!ELEMENT img EMPTY>
  <!ATTLIST img src ENTITY #REQUIRED>
  <!ENTITY logo SYSTEM "logo.gif" NDATA gif>
  <!NOTATION gif PUBLIC "/usr/bin/gimp">
]>
<!-- XML -->
<img src="logo"/>
<vrml>Instructions VRML</vrml>
            \end{lstlisting}
    \section{XML Schema}
        On peut voir le \textbf{XML Schema} ou \textbf{XSD} comme une amélioration de la \textbf{DTD}.
        \subsection{Commentaires}
            \begin{lstlisting}[language=XML]
<xsd:annotation>Com.</xsd:annotation>
            \end{lstlisting}
        \subsection{Base}
            \begin{lstlisting}[language=XML]
<?xml version="1 .0">
<xsd:schema
 xmlns:xsd="http://www.w3.org/2001/XMLSchema"
 targetNamespace="http://www.librairie.org"
 xmlns="http://www.librairie.org">
<!-- code XSD -->
</xsd:schema>
            \end{lstlisting}
            Du fait de sa taille, je ne l'écrirais plus ensuite.

        \subsection{Élément de base}
            \begin{tabular}{| l | l |}
                \hline
                xsd:string & Chaine normale\\
                xsd:normalizedString & Idem que prec. sans $\backslash$r $\backslash$n $\backslash$t\\
                xsd:token & Idem que prec. sans espace.\\
                \hline
                xsd:time & hh:mm:ss\\
                xsd:date & AAAA-MM-JJ\\
                xsd:dateTime & AAAA-MM-JJThh:mm:ss\\
                xsd:duration & PnAnMnJTnHnMnS\\
                \hline
                \multicolumn{2}{| l |}{xsd:int xsd:long xsd:byte xsd:short xsd:decimal}\\
                \hline
                xsd:unsignedXxxxxx & Xxxxxx : Int Short ...\\
                \hline
                xsd:boolean & 1 : true, 0 : false\\
                \hline
                xsd:hexBinary & Données Hexa\\
                \hline
                xsd:anyURI & URL\\
                \hline
            \end{tabular}
            \\
            Le \textbf{T} dans \textit{dateTime} fait la séparation.\\
            Pour \textit{duration}, \textit{nX} : nombre de XXXXX.\\
            On l'utilise ainsi :
            \begin{lstlisting}[language=XML]
<xsd:element name="nom" type="xsd:xxxxxxx"/>
            \end{lstlisting}
            On peut ajouter les attributs suivants aux \textbf{xsd:element} :
            \begin{itemize}
                \item \textbf{maxInclusive}, \textbf{minInclusive},\textbf{maxExclusive}, \textbf{minExclusive} pour encadrer les nombres;
                \item \textbf{length}, \textbf{maxLength}, \textbf{minLength} : Longueur du texte;
                \item \textbf{pattern} : Une regex.
            \end{itemize}
        \subsection{Élément complexe}
            Il faut voir ça comme les éléments souvent basés sur des ensembles :
            \begin{lstlisting}[language=XML]
<xsd:element name="nom">
  <xsd:complexType>
    <xsd:sequence>
      <xsd:element ref="autreNom"/>
      <xsd:element name="autreNom2"
          type="xsd:string"/>
    </xsd:sequence>
  </xsd:complexType>
</xsd:element>  
            \end{lstlisting}
            Sur les \textit{xsd:element} avec l'attribut \textit{ref} peuvent avoir ces attributs :
            \begin{itemize}
                \item \textbf{minOccurs} : unbounded ou un nombre : Nombre minimum d'occurence;
                \item \textbf{maxOccurs} : unbounded ou un nombre : Nombre maximum d'occurence;
            \end{itemize}
            On peut même faire de l'héritage :
            \begin{lstlisting}[language=XML]
<xsd:complexType name="nomFils">
 <xsd:ComplexContent>
  <xsd:extension base="nom">
    <xsd:sequence>
      <xsd:element name="autreNom3"
          type="xsd:string"/>
    </xsd:sequence>
  <xsd:extension>
 </xsd:complexContent>
</xsd:complexType>
            \end{lstlisting}
            Faire un choix :
            \begin{lstlisting}
<xsd:complexType name="pere">
  <xsd:choice minOccurs="0"
      maxOccurs="unbounded">
    <xsd:element name="n1"
        type="xsd:string"/>
    <xsd:element name="n2"
        type="xsd:string"/>
  </xsd:choice>
</xsd:complexType>
            \end{lstlisting}
            \begin{lstlisting}
<!ELEMENT element(n1|n2)*>
            \end{lstlisting}
            On peut aussi restreindre certaines valeurs, ici on redéfinit le maximum d'éléments :
            \begin{lstlisting}
<xsd:ComplexContent>
  <xsd:restriction base="base">
    <xsd:sequence>
      <xsd:element name="elemDeBase"
        type="xsd:integer" maxOccurs="100"/>
    </xsd:sequence>
  <xsd:restriction>
</xsd:ComplexContent>
            \end{lstlisting}
            Pour un \textbf{xsd:string} :
            \begin{lstlisting}
<xsd:restriction base="xsd:string">
  <xsd:enumeration value="value1"/>
  <xsd:enumeration value="value2"/>
</xsd:restriction>
            \end{lstlisting}
            Enfin le \textbf{xsd:complexType} peut recevoir l'argument :
            \begin{itemize}
                \item final=``restriction'' : Restriction impossible;
                \item final=``derivation'' : Dérivation impossible;
                \item final=``\#All'' : Les deux.
            \end{itemize}

        \subsection{Créer un nouveau type}
            On le structure ainsi :
            \begin{lstlisting}
<xsd:simpleType name="nomDuNewType">
  <xsd:restriction base="xsd:typeDeBase">
    <xsd:pattern value="regex"/>
  </xsd:restriction>
</xsd:simpleType>
            \end{lstlisting}
            On doit lui donnée un nom, et un type de base; le plus simple est un String.\\
            Ensuite, on met un regex pour formaliser. On l'utilise :
            \begin{lstlisting}[language=XML]
<xsd:element name="name"
	type="nomDuNewType"/> 
            \end{lstlisting}
            Dans un cas où l'on aurait des bornes de valeurs :
            \begin{lstlisting}
<xsd:simpleType name="valAutorise">
  <xsd:restriction base="xsd:integer">
    <xsd:minInclusive value="-5"
        fixed="true"/>
    <xsd:maxInclusive value="5"/>
  </xsd:restriction>
</xsd:simpleType>
            \end{lstlisting}
        \subsection{Élément vide}
            \begin{lstlisting}
<xsd:element name="rienAvec1Attri">
  <xsd:complexType>
    <xsd:attribute name="attribut1"
        type="xsd:string"/>
  </xsd:complexType>
</xsd:element>
            \end{lstlisting}
    \section{Formes normales}
        \begin{itemize}
            \item \textbf{1NF :} Tous les attributs sont atomiques.
            \item \textbf{2NF :} Un attribut non clé ne dépend pas d'une partie de la clé.
            \item \textbf{3NF :} Un attribut non clé ne dépend pas d'un ou plusieurs attributs ne participant pas à la clé.
            \item \textbf{BCNF :} Tous les attributs non-clé ne sont pas source de dépendance fonctionnelle vers une partie de la clé.
        \end{itemize}


\vfill
\fcolorbox{black}{black}{
\begin{minipage}{\linewidth}
{\color{white}\begin{center}\hfill http://github.com/Servuc/jaizappe \hfill\LaTeX\hfill Licence GPLv3\hfill $\,$\end{center}}
\end{minipage}}
\end{document}