\documentclass[10pt,foldmark,notumble]{leaflet}

\usepackage[utf8]{inputenc}
\usepackage{graphicx}
\usepackage{listingsutf8}
\usepackage{color}
\usepackage[usenames,x11names]{xcolor}
\definecolor{citeGray}{HTML}{696969}
\definecolor{bulletlist}{HTML}{001020}
\definecolor{section}{HTML}{111111}
\definecolor{subsection}{HTML}{222222}
\definecolor{subsubsection}{HTML}{333333}
\usepackage{fourier}
\usepackage[top=0cm,bottom=0cm,left=0cm,right=0cm]{geometry}
\usepackage{caption}
\usepackage{makeidx}
\usepackage{amsmath,amsfonts,amssymb}
\usepackage{adjustbox}
\usepackage{blindtext}
\usepackage[babel=true]{csquotes}
\captionsetup{figurewithin=none}
\captionsetup{tablewithin=none}
\usepackage{changepage}

\def\changemargin#1#2#3#4{\list{}{\parsep#4\topsep#3\rightmargin#2\leftmargin#1}\item[]}
\let\endchangemargin=\endlist

\newcommand{\bmar}{\begin{changemargin}{0.2cm}{0.2cm}{-0.07cm}{0cm} }
\newcommand{\emar}{\end{changemargin}}

\lstset{literate=
  {á}{{\'a}}1 {é}{{\'e}}1 {í}{{\'i}}1 {ó}{{\'o}}1 {ú}{{\'u}}1
  {Á}{{\'A}}1 {É}{{\'E}}1 {Í}{{\'I}}1 {Ó}{{\'O}}1 {Ú}{{\'U}}1
  {à}{{\`a}}1 {è}{{\`e}}1 {ì}{{\`i}}1 {ò}{{\`o}}1 {ù}{{\`u}}1
  {À}{{\`A}}1 {È}{{\'E}}1 {Ì}{{\`I}}1 {Ò}{{\`O}}1 {Ù}{{\`U}}1
  {ä}{{\"a}}1 {ë}{{\"e}}1 {ï}{{\"i}}1 {ö}{{\"o}}1 {ü}{{\"u}}1
  {Ä}{{\"A}}1 {Ë}{{\"E}}1 {Ï}{{\"I}}1 {Ö}{{\"O}}1 {Ü}{{\"U}}1
  {â}{{\^a}}1 {ê}{{\^e}}1 {î}{{\^i}}1 {ô}{{\^o}}1 {û}{{\^u}}1
  {Â}{{\^A}}1 {Ê}{{\^E}}1 {Î}{{\^I}}1 {Ô}{{\^O}}1 {Û}{{\^U}}1
  {œ}{{\oe}}1 {Œ}{{\OE}}1 {æ}{{\ae}}1 {Æ}{{\AE}}1 {ß}{{\ss}}1
  {ç}{{\c c}}1 {Ç}{{\c C}}1 {ø}{{\o}}1 {å}{{\r a}}1 {Å}{{\r A}}1
  {€}{{\EUR}}1 {£}{{\pounds}}1
}

\usepackage{sectsty}
\sectionfont{\color{section}{}}
\subsectionfont{\color{subsection}{}}
\subsubsectionfont{\color{subsubsection}{}}
\definecolor{grisFonce}{HTML}{333333}
\definecolor{grisClair}{HTML}{dddddd}
\lstset{ %
  backgroundcolor=\color{grisClair},
  breaklines=true,
  language=[LaTeX]{TeX}
  }
\usepackage[colorlinks,urlcolor=grisFonce,linkcolor=grisFonce]{hyperref}
\usepackage{titlesec}
\titlespacing*{\section}
  {0.25cm}% decalage a gauche (positif ou negatif)
  {1ex}% espacement vertical avant
  {1ex}% espacement vertical apres
\titlespacing*{\subsection}
  {0.5cm}% decalage a gauche (positif ou negatif)
  {1ex}% espacement vertical avant
  {1ex}% espacement vertical apres
\titlespacing*{\subsubsection}
  {0.75cm}% decalage a gauche (positif ou negatif)
  {1ex}% espacement vertical avant
  {1ex}% espacement vertical apres



% modif 19 decembre 2003
\usepackage[francais]{babel}

%% modif du 25 novemebre 1999
%http://www.tug.dk/FontCatalogue/dejavusans/
%\usepackage[T1]{fontenc}
%\renewcommand*\familydefault{\sfdefault} %% Only if the base font of the document is to be sans serif
\usepackage[light]{kurier}
\usepackage[T1]{fontenc}

\setlength{\parindent}{0pt}
\setlength{\parskip}{0pt}

\title{\#Jaizappé ...}
\author{Servuc}
\date{2016}


\pdfinfo{%
  /Title    (\#Jaizappé ...)
  /Author   (Servuc)
  /Creator  (Servuc)
  /Producer (Servuc)
  /Subject  (Cours)
  /Keywords ()
}

\usepackage{chemfig}

\begin{document}
\fcolorbox{black}{black}{
\begin{minipage}{\linewidth}
\begin{center}
{\Huge{\color{white}\#Jaizappé ...\\... le \LaTeX : Avancé }}
\end{center}
\end{minipage}}
\begin{center}
    \emph{Ce document est la suite du Jaizappé ... le \LaTeX}.
\end{center}

\section{Les mathématiques}
    Il faudra ajouter ceci dans l'entête de votre fichier :
    \begin{lstlisting}
\usepackage{amsmath,amsfonts,amssymb}
    \end{lstlisting}
    \subsection{Bases}
        Une équation est encadrée par deux \$. Exemple, $5+3=8$ :
        \begin{lstlisting}
$ 5 + 3 = 8 $
        \end{lstlisting}
    \subsection{Symbôles}
        Gauche : La commande, Droite : Le résultat. (Liste non ehxaustive)
        \begin{center}
            \begin{tabular}{ l r | l r | l r | l r}
                + & $+$ & - & ${-}$ & \textbackslash times & $\times$ & / & $/$ \\
                \textbackslash$\%$ & $\%$ & $=$ & $=$ & \textbackslash ne & $\ne$ & \textbackslash simeq & $\simeq$ \\
                < & $<$ & \textbackslash leq & $\leq$ & \textbackslash geq & $\geq$ & > & $>$ \\
                \textbackslash prec & $\prec$ & \textbackslash preceq & $\preceq$ & \textbackslash succeq & $\succeq$ & \textbackslash succ & $\succ$ \\
                \textbackslash perp & $\perp$ & \textbackslash in & $\in$ & \textbackslash notin & $\notin$ & \textbackslash ni & $\ni$ \\
                \textbackslash equiv & $\equiv$ & \textbackslash doteq & $\doteq$ & \textbackslash approx & $\approx$ & \textbackslash cong & $\cong$ \\
                \textbackslash exists & $\exists$ & \textbackslash nexists & $\nexists$ & \textbackslash forall & $\forall$ & \textbackslash neg & $\neg$ \\
                \textbackslash land & $\land$ & \textbackslash lor & $\lor$ & \textbackslash emptyset & $\emptyset$ & \textbackslash mapsto & $\mapsto$ \\
                \textbackslash infty & $\infty$ & \textbackslash \{ & $\{$ & \textbackslash backslash & $\backslash$ & \textbackslash \} & $\}$ \\
                \textbackslash ast & $\ast$ & \textbackslash star & $\star$ & \textbackslash cap & $\cap$ & \textbackslash cup & $\cup$ \\
                \textbackslash quad & $\quad$ & \textbackslash dots & $\dots$ & \textbackslash pm & $\pm$ & \textbackslash mp & $\mp$ \\
                \textbackslash vdots & $\vdots$ & \textbackslash ddots & $\ddots$ & \textbackslash cdots & $\cdots$ & \textbackslash ldots & $\ldots$ \\ \\
            \end{tabular}

            \begin{tabular}{ l r | l r | l r }
                \textbackslash subset & $\subset$ & \textbackslash subseteq  & $\subseteq$ & \textbackslash nsubseteq & $\nsubseteq$ \\
                \textbackslash supset & $\supset$ & \textbackslash supseteq  & $\supseteq$ & \textbackslash nsupseteq & $\nsupseteq$ \\
                \textbackslash sqsubset & $\sqsubset$ & \textbackslash sqsubseteq  & $\sqsubseteq$ & \textbackslash parallel & $\parallel$ \\
                \textbackslash sqsupset & $\sqsupset$ & \textbackslash sqsupseteq  & $\sqsupseteq$ & \textbackslash nparallel & $\nparallel$ \\
                \textbackslash sin & $\sin$ & \textbackslash arcsin & $\arcsin$ & \textbackslash sinh & $\sinh$ \\
                \textbackslash cos & $\cos$ & \textbackslash arccos & $\arccos$ & \textbackslash cosh & $\cosh$ \\
                \textbackslash tan & $\tan$ & \textbackslash arctan & $\arctan$ & \textbackslash tanh & $\tanh$ \\
                \textbackslash lim & $\lim$ & \textbackslash ln & $\ln$ & \textbackslash log & $\log$ \\
                \textbackslash min & $\min$ & \textbackslash max & $\max$ & \textbackslash exp & $\exp$ \\
                \\
            \end{tabular}

            \begin{tabular}{ l r | l r }
                \textbackslash sphericalangle & $\sphericalangle$ & \textbackslash measuredangle & $\measuredangle$ \\
                \textbackslash uparrow & $\uparrow$ & \textbackslash downarrow & $\downarrow$ \\
                \textbackslash leftarrow & $\leftarrow$ & \textbackslash rightarrow & $\rightarrow$ \\
                \textbackslash Uparrow & $\Uparrow$ & \textbackslash Downarrow & $\Downarrow$ \\
                \textbackslash Leftarrow & $\Leftarrow$ & \textbackslash Rightarrow & $\Rightarrow$ \\
                \textbackslash Leftrightarrow & $\Leftrightarrow$ & \textbackslash Updownarrow & $\Updownarrow$ \\
            \end{tabular}
        \end{center}
    \subsection{Exposant et indice}
        Un exposant est précédé par $\hat{ }$ et un indice par \_. Si l'indice ou l'exposant sont de plus de 1 caractère, on les entoure de \{xx\}.\\$5^3=a$, $x_a$, $^n/_m$
        \begin{lstlisting}
$ 5^3=a   x_a   ^n/_m $
        \end{lstlisting}
        $x_{12}^b$, $x^3_{a_1}$
        \begin{lstlisting}
$ x_{12}^b     x^3_{a_1} $
        \end{lstlisting}



    \subsection{Fractions et racines}
        Les fractions sur une ligne : $\frac{5}{x + n} \frac{5}{\frac{x + n}{a}}$, sur plusieurs : $\cfrac{5}{\cfrac{x + n}{a}}$
        \begin{lstlisting}
$\frac{5}{x + n} \frac{5}{\frac{x + n}{a}}$
$\cfrac{5}{\cfrac{x + n}{a}}$
        \end{lstlisting}
        Les racines $\sqrt{n} \sqrt[n]{m}$ ($m$ et $n$ peuvent être tout ! : $\sqrt[x^2]{\frac{3}{5_a}}$)
        \begin{lstlisting}
$\sqrt{n}   \sqrt[n]{m}$
        \end{lstlisting}



    \subsection{Lettres spéciales}
        Il suffir d'écrire la lettre grecque pour l'obtenir, exemple $\delta, \Delta, \alpha, \Gamma$
        \begin{lstlisting}
$\delta  \Delta  \alpha  \Gamma$
        \end{lstlisting}

        Certaines lettres ont une 3$\up{ème}$ forme : $\varepsilon \vartheta \varkappa \varpi \varrho \varsigma \varphi$
        \begin{lstlisting}
$ \varepsilon  \vartheta  \varkappa  \varpi
 \varrho  \varsigma  \varphi $
        \end{lstlisting}

        Les ensembles avec : \textbf{\textbackslash mathbb\{X\}}. Exemple, $\mathbb{N}, \mathbb{X}, \mathbb{R}$ :
        \begin{lstlisting}
$\mathbb{N}, \mathbb{X}, \mathbb{R}$
        \end{lstlisting}

        Les lettres grasses et italiques: $\mathbf{A + 5} - b + \mathit{c \times 3}$
        \begin{lstlisting}
$\mathbf{A + 5} - b + \mathit{c \times 3}$
        \end{lstlisting}

        Les lettres calygraphiques : $\mathfrak{A + 5} - b + \mathcal{C \times 3}$
        \begin{lstlisting}
$\mathfrak{A + 5} - b + \mathcal{C \times 3}$
        \end{lstlisting}


    \subsection{Les accents}
        \begin{center}
            \begin{tabular}{l r | l r | l r }
                a' & $a'$ & a'' & $a''$ & \textbackslash grave\{a\} & $\grave{a}$ \\
                a''' & $a'''$ & a'''' & $a''''$ & \textbackslash acute\{a\} & $\acute{a}$ \\
                \textbackslash dot\{a\} & $\dot{a}$ & \textbackslash ddot\{a\} & $\ddot{a}$ & \textbackslash not\{a\} & $\not{a}$ \\
                \textbackslash hat\{a\} & $\hat{a}$ & \textbackslash widehat\{a\} & $\widehat{a}$ & \textbackslash vec\{a\} & $\vec{a}$ \\
                \textbackslash tilde\{a\} & $\tilde{a}$ & \textbackslash widetilde\{a\} & $\widetilde{a}$ & \textbackslash underline\{a\} & $\underline{a}$ \\\\
            \end{tabular}
            \begin{tabular}{l r | l r}
                \textbackslash overrightarrow\{AC\} & $\overrightarrow{AC}$ & \textbackslash overleftarrow\{AC\} & $\overleftarrow{AC}$ \\
                \textbackslash overbrace\{AC\} & $\overbrace{AC}$ & \textbackslash underbrace\{AC\} & $\underbrace{AC}$ \\
            \end{tabular}
        \end{center}




    \subsection{Sommes et intégrales}
        La commande \textbf{\textbackslash displaystyle} permet un autre affichage : $\sum_{i}^{n} - \displaystyle\sum_{i}^{n}$
        \begin{lstlisting}
$\sum_{i}^{n} - \displaystyle\sum_{i}^{n}$
        \end{lstlisting}

        \begin{center}
            \begin{tabular}{l r | l r | l r }
                \textbackslash sum & $\sum$ & \textbackslash int & $\int$ & \textbackslash prod & $\prod$ \\
                \textbackslash bigcup & $\bigcup$ & \textbackslash bigcap & $\bigcap$ & \textbackslash bigsqcup & $\bigsqcup$ \\
                \textbackslash iint & $\iint$ & \textbackslash iiint & $\iiint$ & \textbackslash iiiint & $\iiiint$ \\
                \textbackslash bigoplus & $\bigoplus$ & \textbackslash bigotimes & $\bigotimes$ & \textbackslash bigodot & $\bigodot$ \\
                \textbackslash coprod & $\coprod$ & \textbackslash bigvee & $\bigvee$ & \textbackslash bigwedge & $\bigwedge$ \\
            \end{tabular}
        \end{center}

        On peut superposer si on a plusieurs variables : $\displaystyle\sum_{\substack{i = 1\\j = 1}}^n$
        \begin{lstlisting}
$ \displaystyle\sum_{\substack{i=1\\j=1}}^n $
        \end{lstlisting}



    \subsection{Matrices}
        Les matrices $\rightarrow$ comme les tableaux (\textit{tabular}). $\begin{matrix}a & -b\\-c &d\end{matrix}$
        \begin{lstlisting}
$\begin{matrix}a & b\\c & d\end{matrix}$
        \end{lstlisting}

        Pour ajouter des délimiteurs extérieurs (remplace \textit{matrix}):
        \begin{center}
            \begin{tabular}{l r | l r | l r | l r | l r }
                pmatrix & ( ) & bmatrix & $[ ]$ & Bmatrix & $\{ \}$ \\
            \end{tabular}
            \begin{tabular}{l r | l r }
                vmatrix & | | & Vmatrix & || || \\
            \end{tabular}
        \end{center}
        Ajouter des index sur les lignes et colonnes : $\bordermatrix{ ~ & a & b \cr c & 1 & 0 \cr d & 0 & 1 }$
        \begin{lstlisting}
$\bordermatrix{ ~&a&b \cr c&1&0 \cr d&0&1}$
        \end{lstlisting}





\section{Beamer : Les présentations}
    \subsection{Base}
        Un document \textit{Beamer} est comme un document traditionnel.
        \begin{lstlisting}
\documentclass{beamer}
\begin{document}
    % Contenu
\end{document}
        \end{lstlisting}
    \subsection{Le thême et couleur}
        Il suffit avant le \textit{\textbackslash begin\{document\}} :
        \begin{lstlisting}
\usetheme{Nom du theme}
\usetheme{Nom du coloris}
        \end{lstlisting}

        Les thêmes sont :\\
        \begin{tabular}{l l l l l}
            Antibes & Bergen & Berkeley & Berlin & Ilmenau\\
            Darmstadt & Copenhagen & Frankfurt & Goettingen & Warsaw\\
            Dresden & JuanLesPins & Luebeck & Madrid & Malmoe\\
            Marburg & Montpellier & PaloAlto & Pittsburgh & boxes\\
            Singapore & Szeged & Hannover & Rochester & default\\\\
        \end{tabular}

        Les coloris sont :\\
        \begin{tabular}{l l l l l}
            default & albatross & beaver & beetle & crane\\
            dolphin & dove & fly & lily & orchid\\
            rose & seagull & seahorse & whale & wolverine\\
        \end{tabular}


    \subsection{Diapositive (Frame)}
        Une diapo. est créée avec :
        \begin{lstlisting}
\frame{
    \frametitle{Titre de la diapo}
    Contenu !\pause
    \begin{itemize}
        \item Item A\pause
        \item Item B
    \end{itemize}
}
        \end{lstlisting}

        Le \textit{\textbackslash pause} permet de stopper l'affichage du texte dans une \textit{frame}.\\
        Le contenu s'ajoute comment en \LaTeX $\;$normal.\\
        Seuls les \textit{section} et autres se font hors des \textit{frame} (juste avant).\\
        Suivant les thêmes, le sommaire est placé en haut ou sur le côté.


    \subsection{La première page}
        Il suffit de rendre le para-texte inexistant sur cette diapo.
        \begin{lstlisting}
{
\setbeamertemplate{footline}{} % Efface
\setbeamertemplate{headline}{} % Para-texte
\begin{frame}
    % Contenu
    \titlepage % Pratique
\end{frame}
}
\addtocounter{framenumber}{-1} % Corrige total
        \end{lstlisting}

    \subsection{Un meilleur pied de diapositive}
        Cette commande se place avant \textit{\textbackslash begin\{document\}}
        \begin{lstlisting}
\setbeamertemplate{footline}{
 \hspace*{.5cm}\scriptsize{
   \insertshorttitle$\quad$-$\quad$\insertauthor
   \hspace*{50pt} \hfill \insertframenumber /
   \inserttotalframenumber\hspace{0.6cm}
   } \\ \vspace{9pt} }
        \end{lstlisting}


    \subsection{Boites de contenu}
        Il est possible de mettre en avant des informations grâce à des boites avec des couleurs en accords avec leur importance :
        \begin{lstlisting}
\frame{
    \frametitle{Titre de la diapo}
    \begin{block}{Titre}
    IMPORTANT
    \end{block}

    \begin{alertblock}{Titre}
    ATTENTION
    \end{alertblock}

    \begin{exampleblock}{Titre}
    EXEMPLE
    \end{exampleblock}
}
        \end{lstlisting}

    \subsection{Plusieurs colonnes}
        Il suffit d'utiliser les commandes \textit{\textbackslash columns} et \textit{\textbackslash column} :
        \begin{lstlisting}
\frame{
    \frametitle{Titre de la diapo}
    \begin{columns}[T]
        \begin{column}[T]{5cm}
            % Contenu
        \end{column}
        \begin{column}[T]{5cm}
            % Contenu
        \end{column}
    \end{columns}
}
        \end{lstlisting}

\section{Chimie}
    Très utile au niveau universitaire, simple d'utilisation.\\
    Il suffit d'ajouter le package :
    \begin{lstlisting}
\usepackage{chemfig}
    \end{lstlisting}
    \subsection{Les liaisons}
        Les liaisons sont définis par un ou deux symbôle(s) :
        \begin{center}
            \begin{tabular}{l r | l r}
                \textbackslash chemfig\{A-B\} & \chemfig{A-B} & \textbackslash chemfig\{A=B\} & \chemfig{A=B} \\
                 \textbackslash chemfig\{A>B\} & \chemfig{A>B} & \textbackslash chemfig\{A<B\} & \chemfig{A<B} \\
                \textbackslash chemfig\{A$>:$B\} & \chemfig{A>:B} & \textbackslash chemfig\{A$<:$B\} & \chemfig{A<:B} \\
                \textbackslash chemfig\{A>|B\} & \chemfig{A>|B} & \textbackslash chemfig\{A<|B\} & \chemfig{A<|B} \\
                \textbackslash chemfig\{A$\sim$B\} & \chemfig{A~B} & & \\
            \end{tabular}
        \end{center}


    \subsection{Les angles des liaisons}
        Un angle de liaison est définit par $L[:A]$, $L$ liaison et $A$ angle.\\
        On ajoute une lettre pour la molécules avec $L[:A]E$, $E$ l'élément.
        \begin{center}
            \chemfig{H-[:30]O-[:-30]H}
        \end{center}
        \begin{lstlisting}
\chemfig{H-[:30]O-[:-30]H}
        \end{lstlisting}
        On entoure de ( ) pour indiquer un ensemble :
        \begin{center}
            \chemfig{C(-[:0]N(-[:-45]H)(-[:45]H))(-[:90]H)(-[:180]N(=[:180]O))(-[:270]H)}
        \end{center}
        \begin{lstlisting}
\chemfig{C(-[:0]N(-[:-45]H)(-[:45]H))
    (-[:90]H)(-[:180]N(=[:180]O))(-[:270]H)}
        \end{lstlisting}


    \subsection{Les cycles}
        Il suffit de mettre la molécule principale $\ast$ la taille du cycle, puis les liaisons :
        \begin{center}
            \chemfig{O*3(-O-O-)}
        \end{center}
        \begin{lstlisting}
\chemfig{O*3(-O-O-)}
        \end{lstlisting}



    \subsection{Les ions}
        Il suffit d'ajouter $\hat{ }\{+\}$ ou $\hat{ }\{-\}$ ou $\hat{ }\{\backslash oplus\}$ ou $\hat{ }\{\backslash ominus\}$ après $E$, la molécule.
        \begin{center}
            \chemfig{O^{-}(-[:0]H)}
        \end{center}
        \begin{lstlisting}
\chemfig{O^{-}(-[:0]H)}
        \end{lstlisting}

    \subsection{Représentation de Lewis}
        La syntaxe est : $\backslash\{<1n>\dots<in>,E\}$ avec $1 <= i <= 8$ qui représente les 8 emplacements. Et $E$ la molécule.\\
        Chaque $n$ de $<in>$ peut être $.$ ou $:$ pour $1$ ou $2$ électrons.\\
        Il faut seulement mettre les $<in>$ utiles.
        \begin{center}
            \chemfig{\lewis{2:6:,O}(-[:0]H)(-[:180]H)}
        \end{center}
        \begin{lstlisting}
\chemfig{\lewis{2:6:,O}(-[:0]H)(-[:180]H)}
        \end{lstlisting}

\vfill
\fcolorbox{black}{black}{
\begin{minipage}{\linewidth}
{\color{white}\begin{center}\hfill http://github.com/Servuc/jaizappe \hfill\LaTeX\hfill Licence GPLv3\hfill $\,$\end{center}}
\end{minipage}}
\end{document}
