\documentclass[10pt,foldmark,notumble]{leaflet}

\usepackage[utf8]{inputenc}
\usepackage{graphicx}
\usepackage{listingsutf8}
\usepackage{color}
\usepackage[usenames,x11names]{xcolor}
\definecolor{citeGray}{HTML}{696969}
\definecolor{bulletlist}{HTML}{001020}
\definecolor{section}{HTML}{111111}
\definecolor{subsection}{HTML}{222222}
\definecolor{subsubsection}{HTML}{333333}
\usepackage{fourier}
\usepackage[top=0cm,bottom=0cm,left=0cm,right=0cm]{geometry}
\usepackage{caption}
\usepackage{makeidx}
\usepackage{amsmath,amsfonts,amssymb}
\usepackage{adjustbox}
\usepackage{blindtext}
\usepackage[babel=true]{csquotes}
\captionsetup{figurewithin=none}
\captionsetup{tablewithin=none}
\usepackage{changepage}

\def\changemargin#1#2#3#4{\list{}{\parsep#4\topsep#3\rightmargin#2\leftmargin#1}\item[]}
\let\endchangemargin=\endlist

\newcommand{\bmar}{\begin{changemargin}{0.2cm}{0.2cm}{-0.07cm}{0cm} }
\newcommand{\emar}{\end{changemargin}}

\lstset{literate=
  {á}{{\'a}}1 {é}{{\'e}}1 {í}{{\'i}}1 {ó}{{\'o}}1 {ú}{{\'u}}1
  {Á}{{\'A}}1 {É}{{\'E}}1 {Í}{{\'I}}1 {Ó}{{\'O}}1 {Ú}{{\'U}}1
  {à}{{\`a}}1 {è}{{\`e}}1 {ì}{{\`i}}1 {ò}{{\`o}}1 {ù}{{\`u}}1
  {À}{{\`A}}1 {È}{{\'E}}1 {Ì}{{\`I}}1 {Ò}{{\`O}}1 {Ù}{{\`U}}1
  {ä}{{\"a}}1 {ë}{{\"e}}1 {ï}{{\"i}}1 {ö}{{\"o}}1 {ü}{{\"u}}1
  {Ä}{{\"A}}1 {Ë}{{\"E}}1 {Ï}{{\"I}}1 {Ö}{{\"O}}1 {Ü}{{\"U}}1
  {â}{{\^a}}1 {ê}{{\^e}}1 {î}{{\^i}}1 {ô}{{\^o}}1 {û}{{\^u}}1
  {Â}{{\^A}}1 {Ê}{{\^E}}1 {Î}{{\^I}}1 {Ô}{{\^O}}1 {Û}{{\^U}}1
  {œ}{{\oe}}1 {Œ}{{\OE}}1 {æ}{{\ae}}1 {Æ}{{\AE}}1 {ß}{{\ss}}1
  {ç}{{\c c}}1 {Ç}{{\c C}}1 {ø}{{\o}}1 {å}{{\r a}}1 {Å}{{\r A}}1
  {€}{{\EUR}}1 {£}{{\pounds}}1
}

\usepackage{sectsty}
\sectionfont{\color{section}{}}
\subsectionfont{\color{subsection}{}}
\subsubsectionfont{\color{subsubsection}{}}
\definecolor{grisFonce}{HTML}{333333}
\definecolor{grisClair}{HTML}{dddddd}
\lstset{ %
  backgroundcolor=\color{grisClair},
  breaklines=true,
  language=[LaTeX]{TeX}
  }
\usepackage[colorlinks,urlcolor=grisFonce,linkcolor=grisFonce]{hyperref}
\usepackage{titlesec}
\titlespacing*{\section}
  {0.25cm}% decalage a gauche (positif ou negatif)
  {1ex}% espacement vertical avant
  {1ex}% espacement vertical apres
\titlespacing*{\subsection}
  {0.5cm}% decalage a gauche (positif ou negatif)
  {1ex}% espacement vertical avant
  {1ex}% espacement vertical apres
\titlespacing*{\subsubsection}
  {0.75cm}% decalage a gauche (positif ou negatif)
  {1ex}% espacement vertical avant
  {1ex}% espacement vertical apres



% modif 19 decembre 2003
\usepackage[francais]{babel}

%% modif du 25 novemebre 1999
%http://www.tug.dk/FontCatalogue/dejavusans/
%\usepackage[T1]{fontenc}
%\renewcommand*\familydefault{\sfdefault} %% Only if the base font of the document is to be sans serif
\usepackage[light]{kurier}
\usepackage[T1]{fontenc}

\setlength{\parindent}{0pt}
\setlength{\parskip}{0pt}

\title{\#Jaizappé ...}
\author{Servuc}
\date{2016}


\pdfinfo{%
  /Title    (\#Jaizappé ...)
  /Author   (Servuc)
  /Creator  (Servuc)
  /Producer (Servuc)
  /Subject  (Cours)
  /Keywords ()
}



\begin{document}
    \fcolorbox{black}{black}{
    \begin{minipage}{\linewidth}
        \begin{center}
            {\Huge{\color{white}\#Jaizappé ...\\... le Haskell}}
        \end{center}
    \end{minipage}}
    \section{Histoire}
        \textbf{Haskell} est un langage de programmation fonctionnel créé en 1990. Il est sert au $\lambda$-calcul et au calcul combinatoire. Il s'éxécute avec \textbf{ghci} et des fichiers en $.hs$.
    \section{Bases}
        \subsection{Commentaires}
            Un commentaire est un message caché à l'exécution.
            \begin{lstlisting}[language=Haskell]
-- commentaire sur 1 lignes
{- commentaire sur 1 ou + -}
            \end{lstlisting}
        \subsection{Arithmétique}
            On a les opérateurs : $+ - \times \setminus$ :
            \begin{lstlisting}[language=Haskell]
ghci> 5 * 4 - (3 + 2) / 1
15.0
            \end{lstlisting}
            \begin{lstlisting}[language=Haskell]
succ 5 -- incrémente
pred 5 -- décrémente
            \end{lstlisting}
            \begin{lstlisting}[language=Haskell]
max 5 4 -- retourne 5
min 5 4 -- retourne 4
            \end{lstlisting}
        \subsection{Booléen}
            On a les opérateurs : \textit{True False not \&\& ||} :
            \begin{lstlisting}[language=Haskell]
ghci> True && not ( False || True)
False
            \end{lstlisting}
            Et pour les nombres et chaines : \textit{== /=} :
            \begin{lstlisting}[language=Haskell]
5 /= 4         -- True
5 == 5         -- True
"lol" == "xD"  -- False
            \end{lstlisting}
            Les chaines sont entourés par des doubles quotes !
        \subsection{Charger un fichier}
            Un fichier permet d'omettre les \textbf{let} :
            \begin{lstlisting}[language=Haskell]
:l monFichier    -- Charge le fichier
:reload          -- Recharge le fichier
            \end{lstlisting}
            \section{Fonctions}
        \subsection{Appel d'une fonction}
            \begin{lstlisting}[language=Haskell]
bar 3 "lol" 5.34
bar (3, "lol", 5.34)
            \end{lstlisting}
        \subsection{Définition d'une fonction}
            \begin{lstlisting}[language=Haskell]
carre x = x * x -- fait le carré
func x = if(x > 5) then x - 5 else x + 5
{- Si param x > 5, retourne x - 5
   Sinon retourne x + 5             -}
   
somme x y z = x + y + z
somme 1 2 3 -- 6
            \end{lstlisting}
        \subsection{Fonction avec valeurs prédéfinies}
            \begin{lstlisting}[language=Haskell]
fonc 1 = "Unité"
fonc 10 = "Dizaine"
fonc x = "N/A"
            \end{lstlisting}
        \subsection{Fonction récursive}
            \begin{lstlisting}[language=Haskell]
fonc 0 = 0
fonc x = x + fonc(x - 1)
            \end{lstlisting}
        \subsection{Fonction $\lambda$ (anonyme)}
            Précédé par $\setminus$ et sans nom :
            \begin{lstlisting}[language=Haskell]
carre ((\a -> a + 2) 3) -- 25
            \end{lstlisting}
            Ici, à usage unique, on augmente de 2 le paramètre.
            \begin{lstlisting}[language=Haskell]
(\a b c -> ....) -- usage avec param > 1
            \end{lstlisting}
        \subsection{Fonction $\$$ (joker)}
            \begin{lstlisting}[language=Haskell]
map ($5) [(carre), (*5)] -- [25,25]
            \end{lstlisting}
            Ou sert à calculer les paramètres :
            \begin{lstlisting}[language=Haskell]
length $ [1,2] ++ [3,4] -- 4
            \end{lstlisting}
        \subsection{Composition de fonction}
            L'exemple fait le carré puis la racine :
            \begin{lstlisting}
map (sqrt . carre) [5,3,9]
[5.0,3.0,9.0]
            \end{lstlisting}




    \section{Types}
        \subsection{Base}
            Pour récupérer le type de quelques choses :
            \begin{lstlisting}[language=Haskell]
:t x
            \end{lstlisting}
        \subsection{Définition de type}
            \begin{lstlisting}[language=Haskell]
data NewType = value|value
data Coul = RGB Int Int Int deriving Show
:t RGB -- RGB :: Int -> Int -> Int -> Coul
data Arbre a = ArbreVide |
  Arbre {noeud::a, fg::Arbre a, fd::Arbre a}
  deriving Show
:t Arbre
--Arbre::a -> Arbre a -> Arbre a -> Arbre a
            \end{lstlisting}
            Le dernier cas nous donne les accesseurs directement !\\Sinon, on peut le faire manuellement :
            \begin{lstlisting}
composanteRouge (RGB r _ _) = r
            \end{lstlisting}
        \subsection{Classe de type}
            \begin{lstlisting}
class MonEgal a where
	egal :: a -> a -> Bool
	diff :: a -> a -> Bool
	diff x y = not(egal x y)
	egal x y = not(diff x y)
data MonType = V1|V2
instance MonEgal Coul where
	egal V1 V1 = True
	egal V2 V2 = True
	egal _ _ = True
instance Show MonType where
	show V1 = "valeur 1"
	show V2 = "valeur 2"
            \end{lstlisting}

    \section{Variables et listes}
        \subsection{Bases}
            \begin{lstlisting}[language=Haskell]
let a = 3 -- met 3 dans a
let b = [3, 5, 6] -- met une liste dans b
let c = [] -- liste vide
            \end{lstlisting}
            On peut mettre une liste dans une liste :
            \begin{lstlisting}[language=Haskell]
['l'] ++ ['o', 'l'] -- == "l" ++ "ol"
[4, 3] ++ [2, 1]    -- == [4, 3, 2, 1]
5:[4, 3, 2, 1]      -- optimiser
            \end{lstlisting}
            De même une chaine est une liste :
            \begin{lstlisting}[language=Haskell]
"lol" -- équivaut ['l', 'o', 'l']
'l':"ol" -- pour 1 élément (optimiser)
            \end{lstlisting}
        \subsection{Opérations}
            L'opérateur \textbf{!!} sert à prendre l'élément en X :
            \begin{lstlisting}[language=Haskell]
[4, 3, 2, 1] !! 2 -- retourne 2
            \end{lstlisting}
            Opérations : \textbf{< <= => > == /=},\\
            Suit l'ordre des cases :
            \begin{lstlisting}[language=Haskell]
[3, 2, 1] > [4, 2, 1] -- False
[3, 2, 1] > [3, 2]    -- True
            \end{lstlisting}
            Sélections précises :
            \begin{lstlisting}[language=Haskell]
head [9, 7, 5, 3] -- 9
tail [9, 7, 5, 3] -- [7, 5, 3]
init [9, 7, 5, 3] -- [9, 7, 5]
last [9, 7, 5, 3] -- 3
            \end{lstlisting}
            Fonctions sur les listes :
            \begin{lstlisting}[language=Haskell]
null [] -- True (car pas d'élément)
drop 2 [9, 7, 5, 3]  -- [5, 3]
elem 1 [5, 9, 3, 7]  -- False (non trouvé)
zip [1, 4] "hi"      -- [('h',1),('i',4)]
            \end{lstlisting}
            Il y a aussi \textit{maximum}, \textit{minimum}, \textit{sum}, \textit{product}, \textit{length} et \textit{reverse}.
            Faire des listes préfabriquées :
            \begin{lstlisting}[language=Haskell]
[1..31]       -- De 1 à 31, idem avec lettre
[0..]         -- 0 à infini
[2,4..9]      -- [2,4,6,8]
cycle [1,2]   -- [1,2,1,2,...]
take 5 [1,2]  -- [1,2,1,2,1]
take 2 [9, 7, 5, 3]  -- [9, 7]
replicate 2 4 -- [4,4]
            \end{lstlisting}
            On peut même faire des ensembles :
            \begin{lstlisting}[language=Haskell]
[x+3 | x <- [1..4]] -- [4,5,6,7]
            \end{lstlisting}
            Même avec du code :
            \begin{lstlisting}[language=Haskell]
[if x == 5 then "o" else "L"|x<-[3..7],odd x]
-- ["L", "o", "L"]
            \end{lstlisting}
            On peut faire \textbf{import Data.List} dans le prélude :
            \begin{lstlisting}[language=Haskell]
intersperse 4 [5, 7, 2]-- [5, 4, 7, 4, 2]
--intercalate : idem mais avec listes
transpose [[1,2],[3,4]] -- [[1,3],[2,4]]
concat [[1,2],[3,4]]    -- [1,2,3,4]
            \end{lstlisting}
            Mots clés utiles :
            \begin{lstlisting}[language=Haskell]
and $ map (/=4) [1,2,3]   -- True
or $ map (<4) [1,2,3,4]   -- True
all $ map (/=4) [1,2,3]   -- True
any $ map (==4) [1,2,3]   -- False
            \end{lstlisting}
            Et d'autres fonctions utiles :
            \begin{lstlisting}[language=Haskell]
splitAt 2 [1,2,3,4]   -- ([1,2],[3,4])
splitAt 2 "cool"      -- ("co","ol")
takeWhile (<3) [1,2,3,4] -- [1,2]
dropWhile (<3) [1,2,3,4] -- [3,4]
break (==3) [1,2,3,4] -- ([1,2],[3,4])
span (/=3) [1,2,3,4]  -- ([1,2],[3,4])
inits "lol" -- ["","l","lo","lol"]
tails "lol" -- ["lol","ol","l",""]
            \end{lstlisting}
            Nous avons aussi la calculatrice polonaise :
            \begin{lstlisting}[language=Haskell]
solveRPN "2 3 +" -- 6
            \end{lstlisting}

    \section{Modules}
        Un module contient des fonctions, des types.\\
        Pour importer \textit{Data.List}, deux méthodes :
        \begin{lstlisting}[language=Haskell]
import Data.List   -- Dans fichier
:m + Data.List     -- Dans interpréteur
        \end{lstlisting}
        De même, on peut restreindre l'import (\textit{hiding} = mais pas)
        \begin{lstlisting}[language=Haskell]
import Module.X (fonc1, fonc2)
import Module.X hiding (fonc1, fonc2)
        \end{lstlisting}
    \section{Maybe}
        \textbf{Maybe} fait office d'exception :
        \begin{lstlisting}[language=Haskell]
let f a b = if b == 0 then Nothing
		else Just (div a b)
fmap (\x -> x + 4) (f 4 2)
Just 6
        \end{lstlisting}
    \section{Functor applicatif}
        Il faut importer \textbf{import Control.Applicative},\\
        Type du \textit{functor applicatif} :
        \begin{lstlisting}[language=Haskell]
(<*>)::Applicative f=>f(a -> b)-> f a -> f b 
        \end{lstlisting}
        Exemples :
        \begin{lstlisting}[language=Haskell]
[\x -> x+1, \ x -> x*2] <*> [2 ,4]
    -- [3, 5, 4, 8]
[ ( * ) ] <*> [1, 2, 3] <*> [4, 5]
    -- [4, 5, 8, 10, 12, 15]
        \end{lstlisting}
        Et pour le type \textit{Maybe} :
        \begin{lstlisting}[language=Haskell]
Just ( * 3 ) <*> Just 4 -- Just 12
        \end{lstlisting}
    \section{Les monades}
        Les monades servent à faire des calculs tout en se protégeant des \textit{null}, ces valeurs valent alors \textit{Nothing}.
        \begin{lstlisting}[language=Haskell]
: t (>>=)
(>>=) :: Monad m => m a->( a -> m b )->m b
        \end{lstlisting}
        Exemple :
        \begin{lstlisting}[language=Haskell]
[ 2 .. 4] >>= ( \ x -> [ 0 .. x ] )
--[0 ,1 ,2 ,0 ,1 ,2 ,3 ,0 ,1 ,2 ,3 ,4 ]
Nothing >> Just 4 -- Nothing
Just 3 >> Nothing -- Nothing
Just 3 >> Just 4  -- Just 4
Just 4 >> Just 3  -- Just 3
        \end{lstlisting}
    \section{Les monoïdes}
        C'est un élément munie d'une loi de composition interne ainsi que d'un élément neutre :
        \begin{lstlisting}
import Data.Monoid
Just(Sum 2) `mappend` Just (Sum 3)
-- Just (Sum {getSum = 5})
Just(Sum 2) `mappend` Nothing
-- Just (Sum {getSum = 2})
        \end{lstlisting}
        Une LCI prend en compte que les éléments sous symétriques, commutatifs et associatifs.
        \begin{lstlisting}
Just 2 >>= (\x -> guard(x > 1)) >> Just 3
	-- Just 3
Just 0 >>= (\x -> guard(x > 1)) >> Just 3
	-- Nothing
        \end{lstlisting}
        \textit{Maybe a} est un monoïde si \textit{a} est un monoïde.


\vfill
\fcolorbox{black}{black}{
\begin{minipage}{\linewidth}
{\color{white}\begin{center}\hfill http://github.com/Servuc/jaizappe \hfill\LaTeX\hfill Licence GPLv3\hfill $\,$\end{center}}
\end{minipage}}
\end{document}